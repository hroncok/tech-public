%pdflatex-this
\documentclass[10pt]{article}
\usepackage[czech]{babel}
\usepackage[top=20mm, bottom=15mm, includefoot, left=20mm, right=20mm]{geometry}
\usepackage[utf8]{inputenc}
\usepackage{indentfirst}
\usepackage[unicode=true]{hyperref}
\usepackage[parfill]{parskip}
\usepackage{enumerate}

% římské číslování pro section
\renewcommand \thesection{Čl. \arabic{section}}

\usepackage{fancyhdr}
\pagestyle{fancy}
\fancyhf{}
% jednostranná sazba
\fancyfoot[C]{\thepage}
\lhead{Stanovy klubu tech@su}
\rhead{verze 1.3}

\hypersetup{
	pdfauthor={Přípravný výbor tech@su},
	pdftitle={Stanovy klubu tech@su},
	pdfkeywords={tech@su, tech, SU, Studentká Unie, stanovy},
	colorlinks=true,
	hidelinks=true,
}

\begin{document}

\section{Úvodní ustanovení} %1
	\begin{enumerate}[I.]
	\item Název klubu je tech@su.
	\item Klub tech@su je zájmovým klubem Studentské unie ČVUT (dále jen „SU”).  
	\item Nedílnou součástí stanov tech@su jsou stanovy SU a interní předpisy SU ČVUT.
	\end{enumerate}

\section{Členství} %2
	\begin{enumerate}[I.]
	\item Členem tech@su se může stát každý, kdo splňuje podmínky členství v~SU. O~příjetí člena rozhoduje předseda klubu.
	\item Povinnosti člena klubu tech@su: 
		\begin{enumerate}[a.]
		\item dodržovat stanovy a interní předpisy SU
		\item dodržovat stanovy a interní předpisy tech@su
		\item dodržovat a respektovat usnesení představenstva tech@su
		\item platit členský příspěvek v~souladu s~interními předpisy tech@su
		\end{enumerate}
	\item Člen tech@su má právo:
		\begin{enumerate}[a.]
		\item čerpat výhod plynoucích z~členství
		\item být informován o~činnosti tech@su 
		\item písemně se obracet s~připomínkami, podněty, návrhy a žádostmi na orgány tech@su a obdržet do 32 dnů odpověď
		\item volit a být volen do orgánů tech@su v~souladu se stanovami a interními předpisy tech@su
		\end{enumerate}
	\item Podmínky pro zánik členství jsou definovány ve stanovách SU.  
	\item Členství je automaticky pozastaveno rozhodnutím představenstva na základě porušení stanov tech@su nebo interních předpisů tech@su nebo porušením stanov či interních předpisů SU. V~době pozastavení členství nemá člen nárok na žádné členské výhody.     
	\end{enumerate}

\section{Orgány tech@su} %3
Orgány tech@su jsou:
	\begin{enumerate}[I.]
	\item představenstvo 
	\item předseda  
	\item místopředseda  
	\end{enumerate}

\section{Představenstvo} %4
	\begin{enumerate}[I.]
	\item Výkonnou částí tech@su je představenstvo. Hlavním úkolem představenstva je rozhodovat hlasováním o~klíčových záležitostech a investicích, stanovovat strategii rozvoje klubu tech@su a kontrolovat plnění svých rozhodnutí. 
	\item \label{predstavenstvo_slozeni} Představenstvo tvoří předseda, místopředseda a další zvolení členové. Počet členů představenstva musí být prvočíslo větší než 2.
	\item Představenstvo je schopné usnášení, pokud je na schůzi přítomna nadpoloviční většina všech členů představenstva.
	\item Členem představenstva je člen tech@su, který byl řádně zvolen.
	\item Mandát všech členů představenstva zaniká:
	\begin{itemize}
		\item po uplynutí patnácti měsíců od řádných voleb
		\item pokud složení představenstva nesplňuje bod \ref{predstavenstvo_slozeni} po dobu více než 64 dnů
		\item předáním petice s~nesouhlasem k~výkonu všech členů představenstva s~minimálně 50\% podpisů všech členů tech@su
		\item zvolením nového představenstva
	\end{itemize}

	\item Schůzi představenstva může vyhlásit jakýkoliv člen představenstva. Konání schůze musí oznámit všem členům představenstva s dostatečným předstihem před konáním této schůze.


	\item Dolňovací volby musí být vyhlášeny do 32 dnů od doby, kdy představenstvo porušilo podmínku definovanou v~bodu \ref{predstavenstvo_slozeni}.
	\item V~případě zániku mandátu předsedy je představenstvo povinno do 32 dnů zvolit nového předsedu.
	\item \label{povereni} Každý člen představenstva může pro konkrétní schůzi představenstva pověřit jiného člena tech@su, který není členem představenstva, výkonem svých hlasovacích práv. Pokud se jedná o~stálého zástupce, informuje o~této skutečnosti předsedu, jinak musí zastupovaný člen představenstva emailem nebo osobně sdělit jméno svého zástupce předsedovi a ostatním členům představenstva. Jeden člen tech@su může být pověřen pouze jedním členem představenstva.  
	\item Člen představenstva ztrácí mandát:
		\begin{enumerate}[a.]
		\item pokud se bez omluvy nedostaví on, ani jím pověřený člen dle odst. \ref{povereni}. na tři po sobě jdoucí schůze představenstva
		\item pokud přestane být členem klubu, nebo má pozastavené členství v~klubu po dobu víc než 8 dní
		\item písemnou rezignací do rukou předsedy nebo doručením všem členům představenstva
		\end{enumerate}
	\item Volby do představenstva vyhlašuje volební komise, kterou jmenuje předseda. Volby musí proběhnout do 32 dnů od tohoto usnesení. Průběh voleb upravuje interní předpis.
	\item Představenstvo je povinno zveřejnit každé své usnesení do 32 dnů od jeho schválení.
	\end{enumerate}

\section{Předseda} %5
	\begin{enumerate}[I.]
	\item Předseda řídí klub a zastupuje jej navenek.  
	\item Předsedu jmenuje prezident SU na základě rozhodnutí představenstva.
	\item Představenstvo volí předsedu ze současných členů představenstva.
	\item Předseda podává představenstvu návrh na jmenování místopředsedy.
	\item Předseda se může vzdát funkce, a to:  
		\begin{enumerate}[a.]
		\item prohlášením učiněným osobně na schůzi představenstva
		\item písemně do rukou prezidenta SU.  
		\end{enumerate}
	\item Předseda může být odvolán představenstvem pouze hlasováním a to, vysloví-li se pro návrh alespoň dvě třetiny všech členů představenstva. 
	\item Uvolní-li se post předsedy, přísluší do doby, než bude jmenován nový předseda, výkon jeho funkcí a pravomocí místopředsedovi (ne však déle než 32 dní).
	\end{enumerate}

\section{Místopředseda}
	\begin{enumerate}[I.]
	\item Místopředsedu navrhuje předseda z~řad zvolených členů představenstva. Návrh schvaluje představenstvo.  
	\item Místopředseda se může vzdát funkce, a to:  
		\begin{enumerate}[a.]
		\item prohlášením učiněným osobně na schůzi představenstva
		\item písemně do rukou předsedy  
		\end{enumerate}
	\item Místopředseda může být odvolán:  
		\begin{enumerate}[a.]
		\item předsedou s~uvedením důvodu
		\item představenstvem vysloví-li se pro návrh alespoň dvě třetiny všech členů představenstva
		\end{enumerate}
	\end{enumerate}

\section{Závěrečná, přechodná a zrušující ustanovení}
	\begin{enumerate}[I.]
	\item Tyto stanovy nabývají platnosti po schválení Studentským parlamentem SU ČVUT.
	\item Zveřejněním se rozumí umístění na webové stránky tech@su nebo do veřejně přístupného repozitáře.
	\end{enumerate}


\vspace{10mm}

Schváleno přípravným výborem tech@su ve složení: Jméno1, Jméno2, Jméno2; dne \today

Schváleno na zasedání Parlamentu SU dne:

\vspace{30mm}

\hfill V~Praze dne \today

\hfill Přípravný výbor klubu tech@su

\end{document}
