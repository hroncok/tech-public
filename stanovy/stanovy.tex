%pdflatex-this
\documentclass[10pt]{article}
\usepackage[czech]{babel}
\usepackage[top=20mm, bottom=15mm, includefoot, left=20mm, right=20mm]{geometry}
\usepackage[utf8]{inputenc}
\usepackage{indentfirst}
\usepackage[unicode=true]{hyperref}
\usepackage[parfill]{parskip}
\usepackage{enumerate}

% římské číslování pro section
\renewcommand \thesection{Čl. \arabic{section}}

\usepackage{fancyhdr}
\pagestyle{fancy}
\fancyhf{}
% jednostranná sazba
\fancyfoot[C]{\thepage}
\lhead{Stanovy klubu tech@su}
\rhead{verze 1.1}

\hypersetup{
	pdfauthor={Přípravný výbor tech@su},
	pdftitle={Stanovy klubu tech@su},
	pdfkeywords={tech@su, tech, SU, Studentká Unie, stanovy},
	colorlinks=true,
	hidelinks=true,
}

\begin{document}

\section{Úvodní ustanovení} %1
	\begin{enumerate}[I.]
	\item Název klubu je tech@su.
	\item Klub tech@su je zájmovým klubem Studentské unie ČVUT (dále jen „SU”).  
	\item Nedílnou součástí stanov tech@su jsou stanovy SU a interní předpisy SU ČVUT.
	\end{enumerate}

\section{Členství} %2
	\begin{enumerate}[I.]
	\item Členem tech@su se může stát každý, kdo splňuje podmínky členství v SU. O příjetí člena rozhoduje předseda klubu.
	\item Povinnosti člena klubu tech@su: 
		\begin{enumerate}[a.]
		\item dodržovat stanovy a interní předpisy SU
		\item dodržovat stanovy a interní předpisy tech@su
		\item dodržovat a respektovat usnesení představenstva tech@su
		\item platit členský příspěvek
		\end{enumerate}
	\item Člen tech@su má právo:
		\begin{enumerate}[a.]
		\item požívat výhod plynoucích z členství
		\item být informován o činnosti tech@su 
		\item obracet se s připomínkami, podněty, návrhy a žádostmi na orgány tech@su a obdržet do 30 dnů odpověď
		\item volit a být volen do orgánů tech@su v souladu se stanovami tech@su a interními předpisy
		\end{enumerate}
	\item Podmínky pro zánik členství jsou definovány ve stanovách SU.  
	\item Členství je automaticky pozastaveno nezaplacením členských příspěvků anebo rozhodnutím představenstva na základě porušení stanov tech@su nebo interních předpisů tech@su. V době pozastavení členství nemá člen nárok na žádné členské výhody.     
	\end{enumerate}

\section{Orgány tech@su} %3
	\begin{enumerate}[I.]
	\item představenstvo 
	\item předseda  
	\item místopředseda  
	\end{enumerate}

\section{Představenstvo} %4
	\begin{enumerate}[I.]
	\item Výkonnou částí tech@su je představenstvo. Hlavním úkolem představenstva je rozhodovat hlasováním o klíčových záležitostech a investicích, stanovovat strategii rozvoje klubu tech@su a kontrolovat plnění svých rozhodnutí. 
	\item \label{predstavenstvo_slozeni} Představenstvo tvoří předseda, místopředseda a další zvolení členové. Počet členů představenstva musí být prvočíslo větší než 2.
	\item Představenstvo je usnášeníschopné, pokud je na schůzi přítomna nadpoloviční většina všech členů Představenstva.
	\item Členem představenstva může být člen tech@su, který byl řádně zvolen.
	\item Funkční období členů představenstva je maximálně 1 rok. Funkční období každého člena představenstva začíná jeho zvolením a končí 30. února.
	\item Předseda musí do 30-ti dnů vyhlásit doplňovací volby do Představenstva pokud jeho složení nesplňuje podmínku definovanou v bodě \ref{predstavenstvo_slozeni}.
	\item V případě zániku mandátu Předsedy je Představenstvo povinno do 30 dnů zvolit nového předsedu.
	\item \label{povereni} Každý člen představenstva může pro konkrétní schůzi představenstva pověřit jiného člena tech@su, který není členem představenstva, výkonem svých hlasovacích práv. Pokud se jedná o stálého zástupce, informuje o této skutečnosti předsedu, jinak musí zastupovaný člen představenstva emailem nebo osobně sdělit jméno svého zástupce předsedovi a ostatním členům představenstva. Jeden člen tech@su může být pověřen pouze jedním členem představenstva.  
	\item Člen představenstva ztrácí mandát:
		\begin{enumerate}[a.]
		\item pokud se bez omluvy nedostaví on, ani jím pověřený člen dle odst. \ref{povereni}. na tři po sobě jdoucí schůze představenstva
		\item pokud nebyl zvolen na další volební období
		\item pokud přestane být členem klubu, nebo má pozastavené členství v klubu
		\item písemnou rezignací do rukou předsedy nebo doručením všem členům představenstva
		\end{enumerate}
	\item Řádné volby do představenstva vyhlašuje předseda na základě usnesení představenstva schváleného nadpoloviční většinou všech členů představenstva. Volby musí proběhnout do 30 dnů od tohoto usnesení. Po uplynutí jedenácti měsíců od posledních řádných voleb může volby vyhlásit předseda bez souhlasu představenstva. Průběh voleb upravuje interní předpis.
	\item V případě zániku mandátu všech členů představenstva je předseda povinen vyhlásit řádné volby nejpozději 30 dnů ode dne zániku mandátu představenstva.
	\item Představenstvo je povinno zveřejnit každé své usnesení do 30 dnů od jeho schválení.
	\end{enumerate}

\section{Předseda} %5
	\begin{enumerate}[I.]
	\item Předseda řídí klub a zastupuje jej navenek.  
	\item Předsedu jmenuje prezident SU na základě rozhodnutí představenstva.
	\item Předseda podává představenstvu návrh na jmenování místopředsedy.
	\item Předseda může výkonem svých povinností pověřit jiného člena klubu
	\item Předseda se může vzdát funkce, a to:  
		\begin{enumerate}[a.]
		\item prohlášením učiněným osobně na schůzi představenstva
		\item písemně do rukou prezidenta SU.  
		\end{enumerate}
	\item Předseda může být odvolán představenstvem pouze hlasováním a to, vysloví-li se pro návrh alespoň dvě třetiny všech členů představenstva. 
	\item Uvolní-li se post předsedy, přísluší do doby, než bude jmenován nový předseda, výkon jeho funkcí a pravomocí místopředsedovi (ne však déle než 30 dní).
	\end{enumerate}

\section{Místopředseda}
	\begin{enumerate}[I.]
	\item Místopředseda zastupuje předsedu.  
	\item Místopředsedu navrhuje předseda z řad zvolených členů představenstva. Návrh schvaluje představenstvo.  
	\item Místopředseda se může vzdát funkce, a to:  
		\begin{enumerate}[a.]
		\item prohlášením učiněným osobně na schůzi představenstva,  
		\item písemně do rukou předsedy  
		\end{enumerate}
	\item Místopředseda může být odvolán:  
		\begin{enumerate}[a.]
		\item předsedou s uvedením důvodu
		\item představenstvem vysloví-li se pro návrh alespoň dvě třetiny všech členů představenstva
		\end{enumerate}
	\end{enumerate}

\section{Závěrečná, přechodná a zrušující ustanovení}
	\begin{enumerate}[I.]
	\item Tyto stanovy nabývají platnosti po schválení Studentským parlamentem SU ČVUT.
	\item Zveřejněním se rozumí umístění na webové stránky tech@su nebo do veřejně přístupného repozitáře.
	\end{enumerate}


\vspace{10mm}

Schváleno přípravným výborem tech@su dne:

Schváleno na zasedání Parlamentu SU dne:

\vspace{30mm}

\hfill V Praze dne \today

\hfill Přípravný výbor klubu tech@su

\end{document}
