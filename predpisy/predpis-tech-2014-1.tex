%pdflatex-this
\documentclass[10pt]{article}
\usepackage[czech]{babel}
\usepackage[top=20mm, bottom=15mm, includefoot, left=20mm, right=20mm]{geometry}
\usepackage[utf8]{inputenc}
\usepackage{indentfirst}
\usepackage[unicode=true]{hyperref}
\usepackage[parfill]{parskip}
\usepackage{enumerate}
\usepackage{lastpage}

% římské číslování pro section
\renewcommand \thesection{Čl. \arabic{section}}

\def \Rok {2014}
\def \Cislo {1}
\def \Obsah {Volby do představenstva klubu}
\def \Hash {\input{.pdflatexrun/predpis-tech-\Rok-\Cislo.version}}
\def \Datum {DOPLNIT DATUM} % datum schválení předpisu

\usepackage{fancyhdr}
\pagestyle{fancy}
\fancyhf{}
% jednostranná sazba
\lhead{Interní předpis tech-\Rok-\Cislo~\Obsah}
\lfoot{verze \Hash}
\rfoot{strana \thepage/\pageref{LastPage}}

\hypersetup{
	pdfauthor={Přípravný výbor tech@su},
	pdftitle={Interní předpis tech-\Rok-\Cislo~\Obsah},
	pdfkeywords={tech@su, tech, SU, Studentká Unie, interní předpis},
	colorlinks=true,
	hidelinks=true,
}

\begin{document}

\section{Předmět interního předpisu}
	Tento předpis definuje jak probíhají volby do představenstva klubu tech@su, a to zejména vyhlášení, kandidaturu a vyhodnocení výsledků voleb.

\section{Vyhlášení}
	\begin{enumerate}
		\item Volby se vyhlašují oznámením do e-mailové konference klubu minimálně 14 dní před jejich konáním.
		\item Kandidátky musí být zveřejněny minimálně 2 dny před konáním voleb.
		\item Kandidát nesmí být současně členem volební komise.
		\item Vyhlášení voleb musí obsahovat:
			\begin{itemize}
				\item termín a způsob odevzdání kandidátek
				\item místo a čas konání voleb
				\item způsob hlasování
				\item datum vyhlášení výsledků
				\item počet volených pozic
			\end{itemize}
	\end{enumerate}
	
\section{Kandidatura}
	Kandidatura probíhá odevzdáním kandidátky volební komisi. Platná kandidátka musí obsahovat:
	\begin{itemize}
		\item jméno a příjmení kandidáta
		\item souhlas s kandidaturou do představenstva klubu tech@su ve znění: \uv{S kandidaturou do představenstva klubu tech@su souhlasím.}
		\item podpis kandidáta
		\item je doporučeno, nikoliv vyžadováno, vložit do kandidátky také: 
			\begin{itemize}
				\item e-mailovou adresu
				\item fotografii
				\item identifikaci a otisk PGP klíče
			\end{itemize}
	\end{itemize}

\section{Průběh}
	\begin{enumerate}
		\item Na průběh voleb dohlíží volební komise, kterou jmenuje předseda a skládá se z minimálně 3 členů Studentské Unie ČVUT.
		\item Volby jsou anonymní a tajné.
	\end{enumerate}

\section{Vyhodnocení}
	\begin{enumerate}
		\item Po skončení voleb volební komise sečte hlasy jednotlivých kandidátů.
		\item Kandidáti jsou seřazeni podle počtu získaných hlasů od největšího počtu hlasů po nejnižší. Pokud kandidát získá 0 hlasů, tak je ze seznamu vypuštěn a nemůže být zvolen.
		\item Ze seřazeného seznamu je vybráno prvních $X$ kandidátů, kde $X$ je počet volených pozic.
		\item V případě, že počet kandidátů v seznamu je nižší než $X$, pak je ze seznamu vybrán nejbližší nižší počet kandidátů takový, aby konečný počet členů představenstva splňoval všechny podmínky definované ve stanovách a interních předpisech.
		\item Výsledky jsou zveřejněny do 4 dnů od ukončení voleb.
		\item V případě nejasných výsledků\footnote{Nejasnými výsledky se rozumí především případ, kdy dva kandidáti získají shodný počet hlasů. Volební komisi se doporučuje použít metodu rozhodnutí založenou na náhodném výběru.} rozhoduje volební komise.
	\end{enumerate}

\vspace{10mm}

Schváleno představenstvem tech@su dne: \Datum

\vspace{30mm}

\hfill V Praze dne \Datum

\hfill Představenstvo klubu tech@su

\end{document}
