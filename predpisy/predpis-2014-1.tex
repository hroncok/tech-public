%pdflatex-this
\documentclass[10pt]{article}
\usepackage[czech]{babel}
\usepackage[top=20mm, bottom=15mm, includefoot, left=20mm, right=20mm]{geometry}
\usepackage[utf8]{inputenc}
\usepackage{indentfirst}
\usepackage[unicode=true]{hyperref}
\usepackage[parfill]{parskip}
\usepackage{enumerate}

% římské číslování pro section
\renewcommand \thesection{Čl. \arabic{section}}


\def \Rok {2014}
\def \Cislo {1}
\def \Obsah {Volby do představenstva klubu}

\usepackage{fancyhdr}
\pagestyle{fancy}
\fancyhf{}
% jednostranná sazba
\fancyfoot[C]{\thepage}
\lhead{Interní předpis tech-\Rok-\Cislo~\Obsah}
\rhead{verze 0.1}

\hypersetup{
	pdfauthor={Přípravný výbor tech@su},
	pdftitle={Interní předpis tech-\Rok-\Cislo~\Obsah},
	pdfkeywords={tech@su, tech, SU, Studentká Unie, interní předpis},
	colorlinks=true,
	hidelinks=true,
}

\begin{document}

\section{Předmět interního předpisu}
	Tento předpis definuje jak probíhají volby do představenstva klubu tech@su, a to zejména vyhlášení, kandidaturu a vyhodnocení výsledků voleb.

\section{Vyhlášení}
	\begin{enumerate}
		\item Volby se vyhlašují oznámením do konference tech@su.cvut.cz minimálně 14 dní před jejich konáním.
		\item Kandidátky musí být zveřejněny minimálně 2 dny před konáním voleb.
		\item Kandidát nesmí být současně členem volební komise.
		\item Vyhlášení voleb musí obsahovat:
			\begin{itemize}
				\item místo a čas konání voleb
				\item datum vyhlášení výsledků
				\item termín a způsob odevzdání kandidátek
				\item počet volených pozic
			\end{itemize}
	\end{enumerate}
	
\section{Kandidatura}
	Kandidatura probíhá odevzdáním kandidátky volební komisi. Platná kandidátka musí obsahovat:
	\begin{itemize}
		\item jméno a příjmení kadidáta
		\item souhlas s kandidaturou do představenstva klubu tech@su ve znění: "S kandidaturou do představenstva klubu tech@su souhlasím."
		\item podpis kandidáta
	\end{itemize}

\section{Průběh}
	\begin{enumerate}
		\item Na průběh voleb dohlíží volební komise, kterou jmenuje předseda a skládá se z minimálně 3 členů Studentské Unie ČVUT.
		\item Volby jsou anonymní a tajné, každý člen klubu má jeden hlas.
	\end{enumerate}

\section{Vyhodnocení}
	\begin{enumerate}
		\item Po skončení voleb volební komise sečte hlasy jednotlivých kandidátů.
		\item Výsledky jsou zveřejněny do 4 dnů od ukončení voleb.
		\item V případě nejasných výsledků rozhoduje volební komise.
	\end{enumerate}

\vspace{10mm}

Schváleno představenstvem tech@su dne:

\vspace{30mm}

\hfill V Praze dne \today

\hfill Přípravný výbor klubu tech@su

\end{document}
